\documentclass[notitlepage, 10pt]{article}
\title{Newton's Recurrence}
%\author{{\sc Carl, Sam, Hal} \\ 
%{\small University of Wisconsin--Madison}}
%%\date{March 11, 2004}
\usepackage{amssymb}

\usepackage{amscd} 
\usepackage{hyperref}
\usepackage{graphicx}
\hypersetup{pdfstartview=FitH,pdfauthor={HC},
pdftitle={Newton's Recurrence}}

\newtheorem{theorem}{Theorem}
\newtheorem{lemma}{Lemma}

%% \setlength{\evensidemargin}{0in}
%% \setlength{\oddsidemargin}{0in}
%% \setlength{\topmargin}{0in}
%% \setlength{\textwidth}{450pt} 
%% \setlength{\textheight}{635pt}

\setlength{\textheight}{8.5in}
\setlength{\topmargin}{0.5in}
\setlength{\headheight}{0in}
\setlength{\headsep}{0in}
\setlength{\footskip}{0.5in}

\newcommand{\R}{\mathbb{R}}
\newcommand{\Cstar}{\widetilde{\mathbb{C}}}
\newcommand{\onto}{\rightarrow}
\begin{document}

\maketitle

\begin{abstract}
If Newton's method is repeatedly applied, then something cool happens.
\end{abstract}

\section{Introduction}

Newton's method is an algorithm that allows one to approximate the
roots of a function, starting with a guess.

If $f:\R \onto \R$ is a differentiable function, then Newton's method
maps one approximation to a better approximation:

$$N : x \mapsto x - \frac{f(x)}{f'(x)}.$$

Definition a sequence $\{x\}$ using $N$:
$$x_n = N(x_{n-1})  = x_{n-1} - \frac{f(x_{n-1})}{f'(x_{n-1})}. $$

Instead of evaluating $N(x)$ at each iteration of the method, leave it
in terms of the original guess, $x_0$.  Let $f$ be the quadratic
polynomial $f(x)= a(x-r_1)(x-r_2)$.  Then the first two functions in
the sequence are:

$$x_1 = N(x_0) = \frac{ - x_0^2 + r_1 r_2}{ - 2 x_0 + r_2 + r_1}$$

$$x_2 = N^2(x_0) = \frac
{ - x_0^4 + 6 r_1 r_2 x_0^2 - 4 r_1 r_2^2 x_0 + r_1 r_2^3 - 4 r_1^2
  r_2 x_0 + r_1^3 r_2 + r_1^2 r_2^2 } 
{(2 x_0^2 - 2 x_0 r_2 + r_2^2 - 2 x_0 r_1 + r_1^2)( - 2 x_0 + r_2 +
  r_1)}$$ 

%% $$ x_3 = N^3(x_0) = \frac 
%% { - x_0^8 + 28 r_1 r_2 x_0^6 - 56 r_1 r_2^2 x_0^5 + 70 r_1 r_2^3 x_0^4
%% - 56 r_1 r_2^4 x_0^3 + 28 r_1 r_2^5 x_0^2 - 56 r_1^2 r_2 x_0^5 + 70
%% r_1^3 r_2 x_0^4 - 56 r_1^4 r_2 x_0^3 
%% + 28 r_1^5 r_2 x_0^2 - 8 r_1 r_2^6 x_0 + 70 r_1^2 r_2^2 x_0^4 + r_1
%% r_2^7 + r_1^5 r_2^3 + r_1^3 r_2^5 + r_1^7 r_2 + r_1^2 r_2^6 + r_1^6
%% r_2^2 + r_1^4 r_2^4 
%% - 56 r_1^3 r_2^2 x_0^3 + 28 r_1^4 r_2^2 x_0^2 - 8 r_1^5 r_2^2 x_0 - 56
%% r_1^2 r_2^3 x_0^3 + 28 r_1^3 r_2^3 x_0^2 - 8 r_1^4 r_2^3 x_0 + 28
%% r_1^2 r_2^4 x_0^2 
%% - 8 r_1^2 r_2^5 x_0 - 8 r_1^6 r_2 x_0 - 8 r_1^3 r_2^4 x_0}
%% {(2 x_0^4 - 4 x_0^3 r_2 + 6 x_0^2 r_2^2 - 4 r_2^3 x_0 + r_2^4 - 4
%% x_0^3 r_1 + 6 x_0^2 r_1^2 - 4 x_0 r_1^3 + r_1^4)(2 x_0^2 - 2 x_0 r_2
%% + r_2^2 - 2 x_0 r_1 + r_1^2)( - 2 x_0 + r_2 + r_1)}$$

\begin{theorem}\label{rform}
The $n$th iteration of Newton's Recurrence is 
$$
x_n = N^n(x_0) = \frac
{ \sum\limits_{k=1}^{2^n}(-1)^{2^n} {2^n \choose k} 
\Big( r_1 r_2^{2^n-k} - r_2r_1^{2^n-k} \Big) x_0^k }
{\sum\limits_{k=1}^{2^n}(-1)^{2^n} {2^n \choose k}
\Big( r_2^{2^n-k} - r_1^{2^n-k} \Big) x_0^k}.
$$
\end{theorem}

\begin{theorem}\label{abcform}
If $f(x)= ax^2+bx+c$, then the $n$th iteration of Newton's Recurrence
is
$$
x_n = N^n(x_0) = \frac
{ \sum
\Big( ?? \Big) x_0^k }
{\sum
\Big( ?? \Big) x_0^k}.
$$
\end{theorem}


\section{M\"obius Transformation}

Here, we will prove Theorem \ref{rform}.
Define a M\"obius Transformation $M$ as

$$M : z \mapsto \frac{z-r_1}{z-r_2}$$

We claim that $N(x) = M^{-1}(M(x)^2)$, or that the diagram
\[ 
\begin{CD}\label{...} 
\R @> N_f >> \R \\ 
@ V{M} VV @ VV {M} V \\ 
\Cstar @>> (\cdot)^2 > \Cstar \\ 
\end{CD} 
\] 
commutes.

%Look at $M^{-1}(M(z)^2)$.

%\section{Bleg}

%Here, we will prove Theorem \ref{abcform}.


%\begin{thebibliography}{9}
%\bibitem{}
%\end{thebibliography}

\end{document}


