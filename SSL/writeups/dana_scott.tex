\documentclass[10pt]{article}
\title{The Dana Scott Recurrence}
\author{{\sc Hal Canary} \\
{\small University of Wisconsin--Madison}}
%\date{March 15, 2004}
\usepackage{amssymb}
\usepackage{hyperref}
\usepackage{graphicx}
\hypersetup{pdfstartview=FitH,pdfauthor={},
pdftitle={Newton's Recurrence}}

\newtheorem{theorem}{Theorem}
\newtheorem{lemma}{Lemma}

\setlength{\textwidth}{6.3in}
\setlength{\textheight}{8.7in}
\setlength{\topmargin}{0pt}
\setlength{\headsep}{0pt}
\setlength{\headheight}{0pt}
\setlength{\oddsidemargin}{0pt}
\setlength{\evensidemargin}{0pt}

\newcommand{\R}{\mathbb{R}}
\newcommand{\onto}{\rightarrow}

\begin{document}

\maketitle


%\begin{abstract}
%\end{abstract}

\section{Introduction}

In an article on the Somos Sequence \cite{gale91}, David Gale mentions a
recurrence discovered by Dana Scott,
\begin{equation}\label{scott}
a_n a_{n-4} = a_{n-1} a_{n-3} + a_{n-2} \qquad a_1=a_2=a_3=a_4=1,
\end{equation}
and mentions that there exists a number theoretical proof that it
always gives integers. We will present an alternative proof.

Emilie Hogan found a family of similar recurrences indexed by the
parameter $k$ ($k$ is odd):
\begin{equation}\label{hogan}
a_n a_{n-k} = a_{n-1} a_{n-(k-1)} + a_{n-(k-1)/2} + a_{n-(k+1)/2}.
\end{equation}
Hopefully, (\ref{hogan}) can be approached just like we will approach
(\ref{scott}).

\section{Linearizing}

Equation (\ref{scott}) generates the sequence $\{1,$ $1,$ $1,$ $1,$
$2,$ $3,$ $5, 13,$ $22,$ $41,$ $111,$ $191,$ $361,$ $982,\ldots\}$.
Using a computer, we found that this sequence grows like
$\textrm{O}(C^n)$.  This suggested that the sequence satisfied a
linear recurrence, which we proceeded to find:
\begin{equation}\label{linscott}
0 = a_n - 10~ a_{n-3} + 10~ a_{n-6} - a_{n-9}.
\end{equation}
The characteristic equation of (\ref{linscott}) is
\[
0 = x^9 - 10~ x^6 + 10~ x^3 - 1.
\]
Finding the explicit formula for a linear recurrance is fairly simple,
but time-consuming.  It is better to ask a computer to do it.  The
following code will produce the explicit formula.
\begin{quote}
\begin{verbatim}
#begin Maple code
# Manually enter the characteristic polynomial.
chari := (x) -> x^9 - 10*x^6 + 10*x^3 - 1:
# order of characteristic polynomial.
N:=9:
# quadratic recurrance.
quadrat := proc(n) option remember;
  if n<3 then 1
  else return((quadrat(n-1)*quadrat(n-3)+
    quadrat(n-2))/quadrat(n-4));
  fi;
end:

# Solve for roots of characteristic polynomial.
routes := solve(chari(x),x) :
# Get initial conditions from quadratic.
initial := seq(quadrat(n),n=0..N-1):
# Solve for coefficients.
assign(solve({seq( add( coffs[i]*routes[i]^(j-1) ,i=1..N)
  = initial[j],j=1..N)}, {seq(coffs[i],i=1..N)})):
# Explicit function.
explicit := (n) ->
  simplify(add( coffs[i]*routes[i]^n ,i=1..N)):
# Print out explicit formula.
interface(prettyprint=false):
explicit(n);
#verify that explicit satisfies the quadratic recurrence.
evalb(simplify(explicit(n)*explicit(n-4)) =
 simplify(explicit(n-1)*explicit(n-3)+explicit(n-2)));
#end Maple code
\end{verbatim}
\end{quote}
If the last line of that code returns \emph{true} (which it doesn't),
then we have just proved that the Dana Scott Recurrence is equivilant
to the recurrence given in (\ref{linscott}), and since a linear
recurrance that starts with integers always gives integer, the Dana
Scott Recurrence wil also always give integers.

\section{A Better Proof}

Define the sequence $\{a\}$ recursivly:
\begin{equation}\label{linscott2}
	a_n = 10~ a_{n-3} - 10~ a_{n-6} + a_{n-9}.
\end{equation}
With the intial conditions 
$(a_1{}\ldots{}a_9 ) = (1, 1, 1, 1, 2, 3, 5, 13, 22)$.

We wish to prove by induction that $\{a\}$ is the same as the Dana
Scott recurrance, that it satisifes 
\[     	a_n a_{n-4} - a_{n-1} a_{n-3} - a_{n-2} = 0.	\]

For convienence, let 
\[	\phi(n) := a_n a_{n-4} - a_{n-1} a_{n-3} - a_{n-2}	\]
Assume that $\phi(k)=0$ for $k < n$.  Show that $\phi(n) = 0$.
This gives:
\begin{eqnarray*}
	\phi(n-1) & =& a_{n-1} a_{n-5} - a_{n-2} a_{n-4} - a_{n-3} = 0 \\
	\phi(n-2) & =& a_{n-2} a_{n-6} - a_{n-3} a_{n-5} - a_{n-4} = 0\\
	\phi(n-3) & =& a_{n-3} a_{n-7} - a_{n-4} a_{n-6} - a_{n-5} = 0\\
	\phi(n-4) & =& a_{n-4} a_{n-8} - a_{n-5} a_{n-7} - a_{n-6} = 0\\
	\phi(n-5) & =& a_{n-5} a_{n-9} - a_{n-6} a_{n-8} - a_{n-7} = 0\\
	\phi(n-6) & =& a_{n-6} a_{n-10} - a_{n-7} a_{n-9} - a_{n-8} = 0\\
	\phi(n-7) & =& a_{n-7} a_{n-11} - a_{n-8} a_{n-10} - a_{n-9} = 0\\
	\phi(n-8) & =& a_{n-8} a_{n-12} - a_{n-9} a_{n-11} - a_{n-10} = 0\\
	\phi(n-9) & =& a_{n-9} a_{n-13} - a_{n-10} a_{n-12} - a_{n-11} = 0\\
	\phi(n-10)& =& a_{n-10} a_{n-14} - a_{n-11} a_{n-13} - a_{n-12} = 0\\
	& \ldots &
\end{eqnarray*}


Now, compute $\phi(n)$.
\[	\phi(n) = a_{n}  a_{n-4} -  a_{n-1}  a_{n-3} - a_{n-2} \]
Substitute for $a_{n}$ and $a_{n-1}$ from the definition of $\{a\}$.
\begin{eqnarray*}
	a_{n}   &=& 10  a_{n-3} - 10  a_{n-6} + a_{n-9} \\
	a_{n-1}	&=& 10  a_{n-4} - 10  a_{n-7} + a_{n-10} 
\end{eqnarray*}
\begin{eqnarray*}
\phi(n) &=&   ( 10  a_{n-3} - 10  a_{n-6} + a_{n-9}  )  a_{n-4}\\
        && - ( 10  a_{n-4} - 10  a_{n-7} + a_{n-10} )   a_{n-3} \\
	&& - ( 10  a_{n-5} - 10  a_{n-8} + a_{n-11} ) 
\end{eqnarray*}
Simplify:
\begin{eqnarray*}
	\phi(n) &=&  10 a_{n - 3} a_{n - 7} - 10 a_{n - 4} a_{n - 6} - 10 a_{n - 5} \\
		&&+ a_{n - 4} a_{n - 9} - a_{n - 3} a_{n - 10} + 10 a_{n - 8} - a_{n - 11}
\end{eqnarray*}
Since $\phi(n-3)=0$, 
\begin{eqnarray*}
	\phi(n) &=&  a_{n - 4} a_{n - 9}  - a_{n - 3} a_{n - 10}\\ 
		&& + 10 a_{n - 8}  - a_{n - 11}
\end{eqnarray*}

Substitute for $a_{n-3}$ and $a_{n-4}$ from the definition of $\{a\}$.
\begin{eqnarray*}
	a_{n-3} &=& 10  a_{n-6} - 10  a_{n-9}  + a_{n-12} \\
	a_{n-4} &=& 10  a_{n-7} - 10  a_{n-10} + a_{n-13}\end{eqnarray*}
\begin{eqnarray*}
	\phi(n) &=& (10  a_{n-7} - 10  a_{n-10} + a_{n-13} )  a_{n - 9} \\
	        && - (10  a_{n-6} - 10  a_{n-9}  + a_{n-12} ) a_{n - 10} \\
	        && + 10  a_{n - 8} \\
	        && - a_{n - 11} ;
\end{eqnarray*}
Simplify:
\begin{eqnarray*}
	\phi(n) &=& - 10 a_{n - 10}  a_{n - 6}  + 10 a_{n - 9}  a_{n -
		7}  + 10 a_{n - 8} \\ 
		&&  + a_{n - 9}  a_{n - 13} - a_{n - 10}  a_{n - 12} -
		a_{n - 11} 
\end{eqnarray*}
\begin{eqnarray*}
	\phi(n) &=& - 10  ( + a_{n - 10}  a_{n - 6} - a_{n - 9}  a_{n - 7} 
		  - a_{n - 8} )\\
		&& + a_{n - 9}  a_{n - 13} - a_{n - 10}  a_{n - 12} - a_{n - 11}
\end{eqnarray*}
Since $\phi(n-6) = \phi(n-9) = 0$
\[	a_{n-6} a_{n-10} - a_{n-7} a_{n-9} - a_{n-8} = 0 \]
\[	a_{n-9} a_{n-13} - a_{n-10} a_{n-12} - a_{n-11} = 0 \]
\[	\phi(n) = 0 \]
QED.

\section{Aside: Laurent Polynomials}

If we choose the first four terms of (\ref{scott}) to be $(w,x,y,z)$
instead of $(1,1,1,1)$, then the next five terms are
\[a_5=\frac{z x + y}{w}\]
\[a_6=\frac{yzx+y^2+zw}{wx}\]
\[a_7=\frac{yz^2x+zy^2+z^2w+zx^2+xy}{wxy}\]
\[a_8=\frac{yz^3x^2+2y^2z^2x+zy^3+z^3wx+z^2wy+z^2x^3+2zx^2y+xy^2+wy^2zx+wy^3+w^2yz}
{w^2xyz}\]
\[a_9=\frac{
\begin{array}{c}
yz^2x^3+x^2y^2z^3+x^2z^2w+2x^2zy^2+x^2zw^2+2xwyz^3+2xz^2y^3\\
+xz^2w^2y+xzwy+xzwy^3+xy^3+w^2z^3+2z^2y^2w+z^2w^3+2zw^2y^2+zy^4+wy^4
\end{array}
}{x^2yzw^2}\]
These are all Laurent polynomials, so if we choose the first nine
terms of $\{a\}$ to be $(w,x,y,z,a_5,a_6,a_7,a_8,a_9)$
instead of $(1, 1, 1, 1, 2, 3, 5, 13, 22)$, then by the
linearity of (\ref{linscott2}), all subsequent terms are Laurent
polynomials. 

\begin{quote}
\begin{verbatim}
#begin Maple code
a := proc(n)
  if n = 1 then w
  elif n = 2 then x
  elif n = 3 then y
  elif n = 4 then z
  else simplify((a(n - 1)*a(n - 3) + a(n - 2))/a(n - 4))
  end if
end proc:
interface(prettyprint=false);
seq(a(n),n=1..9);
#end Maple code
\end{verbatim}
\end{quote}


\begin{thebibliography}{9}

\bibitem{gale91} 
{\sc David Gale.}  ``The Strange and Surprising Saga of the Somos
Sequences.''  \emph{Mathematical Intelligencer} \textbf{13} 1 (1991)
40-42.

\end{thebibliography}

\end{document}


