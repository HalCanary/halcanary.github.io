\documentclass[notitlepage, 10pt]{article}
\title{Newton's Method as a Formal Recurrance}
%\title{Newton's Method Applied Symbolically to Quadratics}
%\title{Newton's Recurrence}
\author{{\sc 
	Carl Edquist,
	Sam Lachterman,
	Brendan Younger,
	Hal Canary} \\
{\small University of Wisconsin--Madison}}
%\date{April 28, 2004}
\usepackage{amssymb, amsthm, amsmath, amscd}
\usepackage{fullpage}
\usepackage{hyperref}
\usepackage{graphicx}
\hypersetup{pdfstartview=FitH,pdfauthor={SSL},
pdftitle={Newton's Method Applied Symbolically to Quadratics}}

\newtheorem{theorem}{Theorem}
\newtheorem{lemma}{Lemma}
\newtheorem{conjecture}{Conjecture}

\newcommand{\R}{\mathbb{R}}
\newcommand{\Cstar}{\widetilde{\mathbb{C}}}
\newcommand{\onto}{\rightarrow}
\newcommand{\binomial}[2]{\genfrac{(}{)}{0pt}{}{ #1 }{ #2 }}
\newcommand{\qbinomial}[2]{\genfrac{[}{]}{0pt}{}{ #1 }{ #2 }_q }

\begin{document}

\bibliographystyle{plain} % Entries are ordered alphabetically;
%\bibliographystyle{unsrt} % Entries are not ordered alphabetically,
			  %  but in the order they are first referenced.
%\bibliographystyle{abbrv} % The bibliography looks the same as for
			  %  plain style except that first names and
			  %  names of journals and months are abbreviated;
%\bibliographystyle{alpha} % The bibliography looks the same as for
			  %  plain style except that the reference
			  %  markers are not just 1,2,3... but are
			  %  based on authors' initials and publication year;
\bibliographystyle{plain} % Entries are ordered alphabetically;

\maketitle

%% \begin{abstract}
%% Iterating Newton's method symbolically for the general quadratic
%% $ax^2+bx+c$ yields a rational function $\frac{P_n(x)}{Q_n(x)}$, the
%% numerator and denominator of which are polynomials with highly
%% composite coefficients. In particular, the coefficients have no prime
%% factors greater than $2^n$ after $n$ iterations.
%% \end{abstract}

\section*{Non-commuting Algebra}

We have defined $P_n$ and $Q_n$ with $P_0(x)=x$ and $Q_0(x)=1$.  If we
instead let $P_0(x)=x$ and $Q_0(x)=y$, it can be verified that we get
slightly different formula:

\[
P_n(x,y) = a^{2^n-1}x^{2^n}~+~\sum\limits_{k=0}^{(2^n-2)} 
\sum\limits_{i=1}^{~(2^n-k-1)~} 
(-1)^{i} \binomial{2^n }{ k} \binomial{2^n-k-i-1}{i-1}  
a^{k+i-1}~ b^{2^n-k-2i} ~c^i ~x^k y^{2^n-k}
\]
\[
Q_n(x,y) = \sum\limits_{k=0}^{(2^n-1)}
\sum\limits_{i=0}^{~(2^n-k-1)~} (-1)^i \binomial{2^n}{k} 
\binomial{2^n-k-i-1}{i} 
a^{k+i} ~b^{2^n-k-2i-1} ~c^i ~x^k y^{2^n-k}
\]

But suppose that $x$ and $y$ do not commute, but rather satisfy the
formula $yx=qxy$.  What happens?  If we define
\[ \qbinomial{n }{ k} = \prod_{i=1}^{n-k}
\frac{ 1-q^{i+k} }{ 1-q^i }\]
then the q-version of the binomial formula is: \cite{qbin}
\[
(x + y)^n = \sum_{k=0}^{n} \qbinomial{n }{ k} x^k y^{n-k}.
\]

\begin{conjecture}
If $yx=qxy$ and if $P_n(x,y)$ and $P_n(x,y)$ are defined by
\begin{eqnarray*}
& P_0     = x \qquad Q_0 = y \\
& P_{n+1} = a P_n^2 - c Q_n^2  \qquad
Q_{n+1} = a P_n Q_n + a Q_n P_n + b Q_n Q_n .
\end{eqnarray*}
then
\[
P_n(x,y) = a^{2^n-1}x^{2^n}~+~\sum\limits_{k=0}^{(2^n-2)} 
\sum\limits_{i=1}^{~(2^n-k-1)~}  (-1)^{i} 
\qbinomial{2^n }{ k} \binomial{2^n-k-i-1}{i-1}  
a^{k+i-1}~ b^{2^n-k-2i} ~c^i ~x^k y^{2^n-k}
\]
\[
Q_n(x,y) = \sum\limits_{k=0}^{(2^n-1)}
\sum\limits_{i=0}^{~(2^n-k-1)~} (-1)^i 
\qbinomial{2^n }{ k} \binomial{2^n-k-i-1}{i} 
a^{k+i} ~b^{2^n-k-2i-1} ~c^i ~x^k y^{2^n-k}.
\]
\end{conjecture}

It is not clear to us that the proof we provided for Theorem
(\ref{thm:maintheorem}) is aplicable to this more general conjecture.

%% \begin{thebibliography}{9}
%% \bibitem{qbin}
%% {\sc M.~P.~Schutzenberger}. 
%% ``Une interpretation de certaines solutions de l'equation
%% fonctionnelle: $F(x + y) = F(x)F(y)$.''
%% \emph{C.~R.~Acad.~Sci.~Paris}, 236 (1953), 352-353.
%% \end{thebibliography}

\bibliography{newton-with-a-vengance} 
\end{document}


